% GEDEX CD-ROM Booklet for Distribution Package
%
% Written Dec. 1991 by A. Warnock, Hughes STX
%
\documentstyle[twoside]{article}
\textheight 3.8in
\textwidth 4.0in
\oddsidemargin .7in
\evensidemargin .7in
\raggedbottom
%
% Quire input and settings (see quire.doc for more information)
\input quire
\latexquire
%\htotal 5.0in       \vtotal 5.0in
\htotal 4.75in       \vtotal 4.7in
\horigin -0.3in     \vorigin -0.2in
\shoutline 0pt      \shstaplewidth 0pt
\shcrop 1pt
%\shonly2
% Use these two lines to format one booklet page per sheet with crop marks
\shhtotal 8.5in
\qonepage
% Use these two lines to format two booklet pages per sheet with crop marks
% (Make sure the printer is set for landscape mode, not portrait mode)
%\shhtotal 11in
%\quire{16}
%
\pagestyle{myheadings}
\markboth{NASA Climate Data System}
{GEDEX CD-ROM}
%
\begin{document}
\setcounter{page}{0}
\thispagestyle{empty}
\begin{centering}
{\Large\bf GREENHOUSE EFFECT \\
DETECTION EXPERIMENT\\
\bigskip
GEDEX \\
\bigskip
Selected Data Sets} \\
%\bigskip\bigskip\bigskip
\medskip\medskip
Lola M. Olsen  \\
Archibald Warnock III \\
%\bigskip\bigskip\bigskip\bigskip
\bigskip
Prepared as a NASA contribution to \\
The Space Agency Forum \\
on \\
The International Space Year 1992\\
\medskip
by \\
NASA Climate Data System Staff\\
Goddard Distributed Active Archive Center\\
NASA --- Goddard Space Flight Center \\
Greenbelt, Maryland 20771 \\
%\bigskip\bigskip
%Prepared by: \\
%Hughes STX \\
%4400 Forbes Blvd. \\
%Lanham, MD 20706 \\
\end{centering}

\pagebreak
\setcounter{page}{0}
\thispagestyle{empty}
\noindent

\small
\vspace{.5in}
\noindent The data contained herein are for scientific use only and 
have no commercial value.

\vspace{.5in}
\noindent MS-DOS is a registered trademark of Microsoft Corporation. \\
Macintosh is a registered trademark of Apple Computer. \\
UNIX is a registered trademark of AT\&T. \\
VMS is a registered trademark of Digital Equipment Corp.

\pagebreak
\setcounter{page}{1}

\begin{centering}
\Large \bf
The Greenhouse Effect Detection Experiment \\
\bigskip
Selected Data Sets \\
for \\
Global Change Research \\
\end{centering}

\small
\vspace*{3ex}

\section*{Introduction}

In preparation for the International Space Year (ISY), the Greenhouse
Effect Detection Experiment (GEDEX), a NASA Earth Science and
Applications Division-sponsored initiative in support of the Space
Agency Forum on the International Space Year (SAFISY), organized a
workshop to bring together a core group of scientists to share their
research and ideas in the area of global climate change. Participants in
this workshop, which was designated GEDEX Atmospheric Temperature
Workshop met in Columbia, Maryland, in July of 1991 for the purpose of
obtaining a measure of progress and recommending actions required to
better understand the global atmospheric temperature record and its
relationship with climate forcings and feedbacks.  Dr.~Robert
A.~Schiffer and Dr.~Sushel Unninayar organized and led the discussions
where concepts and hypotheses were exchanged.  The document, ``The
Detection of Climate Change Due To The Enhanced Greenhouse Effect: A
Synthesis of Findings Based on the GEDEX Atmospheric Temperature
Workshop,'' issued by NASA Headquarters in February 1992 summarizes the
ideas and discussions which took place during the workshop.

One of the primary objectives of the workshop was to assemble and
document existing data (focusing on temperature) for the analysis of
global climate change and to consolidate these selected data sets onto
CD-ROMs for distribution nationally and internationally to promote
further research. With climate as the focus, Dr.~Schiffer requested that
NASA's Climate Data System (NCDS) staff participate and prepare for the
acquisition, archiving, implementation, and documentation of data
recommended for distribution.  NCDS has been designated as the core
system for the Goddard Space Flight Center's Distributed Active Archive
Center (DAAC) of the Earth Observing System Data and Information System
(EOSDIS), and in this role will continue to update GEDEX-relevant data
sets.  All data will remain in the Goddard DAAC (NCDS) and will be
updated whenever new releases are made available. Subsequent updates
will therefore be made available in an ongoing manner to the climate
user community.  More than 60 data sets were identified by workshop
participants for inclusion, yielding nearly 1 gigabyte of data for this
first 2-disk set of CD-ROMs.

Immediately following the workshop, staff members began gathering and
implementing the designated data to make it available to the climate
user community through the online interactive data system, as well as
preparing the data for publication on CD-ROM.  NCDS staff members also
translated data into a standard format (the Common Data Format [CDF]) to
allow for the use of a single set of software tools to access the data
on disks.

Most participants contributed data and helped in the preparation of the
standard documentation for each data set slated for CD-ROM. Each data
set was verified by the NCDS staff after it was transformed into CDF.
Iterations of the detailed documentation and extensive verification with
data producers ensure that the data are reproduced as received from data
producers.  The data producers cooperated fully with this essential
effort.

The data sets include surface, upper air, and/or satellite-derived
measurements of temperature, solar irradiance, clouds, greenhouse gases,
fluxes, albedo, aerosols, ozone, and water vapor, along with Southern
Oscillation Indices and Quasi-Biennial Oscillation statistics.  Many of
the data sets provide global coverage.  The spatial resolutions vary
from zonal to 2.5 degree grids. Temporal coverage also varies. Some
surface station data sets cover more than 100 years, while most of the
satellite-derived data sets cover only the most recent 12 years.
Temporal resolution, for most data sets, is monthly.

The data sets, thoroughly documented through standard detailed catalogs,
are easily identified through the use of summaries providing temporal
coverage and resolution, spatial coverage and resolution, parameters,
{\em etc\/}.

\section*{GEDEX Data Sets}

%Within the workshop environment, participants recommended data sets that
During the workshop, participants recommended data sets that
they considered directly or indirectly relevant to the study of the
Greenhouse Effect.  One session at the workshop was devoted to
identifying data sets for the CD-ROMs, including recommended temporal and
spatial resolutions. Although the focus of the first GEDEX workshop was
on temperature, other parameters such as solar irradiance, atmospheric
constituents, cloud, and radiation budget data which affect the
temperature record were considered essential components of these disks.
The data on the disks have been categorized by these parameters for
summary here. The temperature, solar irradiance, cloud, and radiation
budget data can be found on Disk 1. The atmospheric constituent data are
contained on Disk 2. Satellite-based data overlap these categories but
are also designated separately. Additionally, the documentation of data
sets is discussed briefly.


\subsection*{Disk 1 --- Temperature, Radiation and Cloud Data}

\subsubsection*{Temperature --- Surface}

The basic surface station temperature data set from NCDC/NCAR contains
monthly temperature and precipitation values and is subdivided by
continent.  A few records date from as early as 1738, and modern station
data extend through 1989. Other surface temperature anomaly data sets
containing monthly gridded values were provided by Philip Jones,
University of East Anglia Climate Research Unit, and by James Hansen,
Goddard Institute for Space Studies (GISS). Zonal and station
temperature data are included from the State Hydrologic Institute's
(Russia) Konstantin Vinnikov. These data sets extend over 100 years of
record.  Gridded 2.5 degree monthly sea surface temperature data and
anomalies as calculated by Richard Reynolds from NOAA's Climate Analysis
Center also reside on this disk. These SST values are from AVHRR sensors
on NOAA polar orbiters and are blended with ship and buoy data.
Investigating the effect of the El Ni\~{n}o/Southern Oscillation (ENSO)
on the temperature anomaly record, may be done with the data set
provided by the University of East Anglia's Climate Research Unit
containing the Southern Oscillation Index calculations, along with the
Tahiti and Darwin mean sea level pressures from which they are derived.


\subsubsection*{Temperature --- Upper Air}

NCDC/NCAR contributed comprehensive monthly station rawinsonde data.
Both temperature and humidity profiles are included in this data set.
Another upper air temperature data set was produced by James Angell,
NOAA ARL.  It contains seasonal zonal temperature deviations from
rawinsonde data around the world. Angell also provided Quasi-Biennial
Oscillation temperature and zonal wind data at 50, 30, and 10 mb.
Marshall Space Flight Center's Roy Spencer provided more than 12 years
of mid-tropospheric temperature and anomaly data from the TIROS
Operational Vertical Sounder Microwave Sounding Unit (TOVS-MSU), flown
on NOAA polar orbiters.  Stratospheric temperature data were provided by
Harry van Loon and Karen Labitzke through NCAR.   Although these data
are only available for the northern hemisphere, they provide a valuable
monthly zonal product for the years 1957 to 1991.  In addition, profiles
of meteorological data from NMC were provided at 1 km intervals for the
Stratospheric Aerosol and Gas Experiment (SAGE II) time period.


\subsubsection*{Solar Irradiance and Transmission}

Solar transmission and surface-measured irradiance data were sent by
Ellsworth Dutton, NOAA Climate Monitoring and Diagnostics Laboratory
(CMDL).  The daily solar transmission indices from the Mauna Loa
Observatory begin in 1958 and continue through 1990. The hourly solar
irradiance data make up a rare collection of solar data collected at the
surface from 1976 to 1989 at selected sites.  NASA Goddard Space Flight
Center's Lee Kyle provided solar irradiance data from the Nimbus-7 Earth
Radiation Budget (ERB) instrument, and Langley Research Center's Robert
Lee, offered the solar irradiance data from NOAA-9, NOAA-10, and ERBS.
Richard Willson of JPL has collaborated with the NCDS staff over the
years in making 9 years of solar irradiance data from the Solar Maximum
Mission's ACRIM sensor available to users online.  The Dominion Radio
Astrophysical Observatory (DRAO) (formerly Ottawa) 2800 MHz radio flux
data from 1947 to the present are also available on the disk with
observed, absolute, and adjusted variables.

\subsubsection*{Radiation Budget and Clouds}

Bruce Barkstrom of Langley Research Center provided the combined
Earth Radiation Budget Experiment's (ERBE S4) satellite gridded
products, including the scanner data at 2.5 degree resolution and the
wide-field-of-view monthly averages. William Rossow, NASA GISS,
suggested and subsequently provided a comprehensive subset of the
International Satellite Cloud Climatology Project's (ISCCP) monthly
cloud products at 2.5 degree resolution.  He also assisted in the review
and verification of those data.  Goddard's Lee Kyle worked closely with
the staff in the validation of data on the disk from the Earth Radiation
Budget instrument on board Nimbus-7.  Data from the wide-field-of-view
sensor span the period 1978 to 1987 and are monthly in temporal
resolution and approximately 4.5 by 5 degrees in spatial resolution.
Goddard's Joel Susskind also worked closely with the NCDS staff, making
subsets of his cloud and radiation data available for the disk.  His
data are derived from NOAA Polar Orbiting satellites using TOVS-HIRS and
TOVS-MSU sensors.


\subsection*{Disk 2 --- Atmospheric Constituents}

\subsubsection*{Atmospheric Constituents}

The Carbon Dioxide Information Analysis Center (CDIAC), Department of
Energy, is the source for the ``TRENDS '90, A Compendium of Data on
Global Change,'' providing carbon dioxide and methane values spanning
the geological record (through ice core techniques) and more recent
values collected by NOAA from flask sampling and continuous monitoring
techniques.  NOAA ARL's James Angell also contributed seasonal layer
ozone data from Umkehr sounding and ozonesonde from 1957 to 1990, and
total ozone from Dobson spectrophotometers for the period 1967 to 1989.
Patrick McCormick's colleagues at NASA's Langley Research Center worked
closely with our staff in providing ozone, nitrogen dioxide, and aerosol
data from the Atmospheric Explorer Mission's SAGE I instrument, and
aerosol, ozone, water vapor, and nitrogen dioxide data from the Earth
Radiation Budget Satellite's (ERBS) SAGE II instrument beginning with
data from the November 1984 launch through 1991.



\subsubsection*{Satellite-based Data}

The satellite-based data sets are those from Solar Maximum Mission's
ACRIM instrument, the Nimbus-7 Earth Radiation Budget instrument, the
Atmospheric Explorer Mission's SAGE I sensor, the Earth Radiation Budget
Satellite's SAGE II sensor, the NOAA polar orbiter TOVS-HIRS and
TOVS-MSU instrument, the Earth Radiation Budget Experiment (NOAA-9,
NOAA-10, and ERBS sensors), and the International Satellite Cloud
Climatology Project's AVHRR, MIR and VAS sensors on NOAA Polar Orbiters,
GOES, Meteosat, and GMS satellites.


\subsection*{Documentation}

Each data set is accompanied by a detailed catalog in a standard format
(in the subdirectory DETAILED) and by a short summary file (in the
subdirectory SUMMARY). In addition, the DETAILED subdirectory also
contains descriptions of satellite sensors and the products from which
the geophysical parameters on this disk are derived.  Among these are
AVHRR, TOVS, VAS, VISSR, ERB, ERBE, SAGE I, and SAGE II.

\section*{The CD-ROM Contents}

For the purposes of this discussion, we assume that the reader has
access to a CD-ROM drive and is familiar with its operation. Drives are
available for nearly every computer and operating system, so a tutorial
in their installation and operation is beyond the scope of this booklet.

All of the actual data on the CD-ROMs are in CDF format, written under
version 2.1 of the CDF library.  The single-file and network-encoding
options were used to make the data most widely portable.  There are also
supplemental text files which provide additional information about the
contents of the data files. In this booklet, the term {\em text file}
implies that the file consists of stream-oriented records delimited by a
carriage-return (decimal value 13) and a linefeed (decimal value 10).
This format is consistent with the MS-DOS$^{\rm TM}$ operating system.
Users of Apple Macintosh$^{\rm TM}$ computers, DEC VAX$^{\rm TM}$  or
UNIX$^{\rm TM}$ workstations may find that one or the other of these
delimiters is not used by the operating system and is ``visible'' in the
file. Regardless, the files should be easily readable by normal text
processing tools such as editors and file browsers.

File names on the discs consist of a name no longer than eight
characters, followed by a period, ``.'', and an extension no longer than
three characters. On Macintosh and some UNIX computers, the full
filename may also contain a trailing ``version number'' separated by a
semi-colon. The file names will appear in upper case under those
operating systems which are case-sensitive.

The directory hierarchy is the same for both CD-ROMs.  There are 5 major
subdirectories on the discs, as given below:

\begin{description}
\item[DATA] The data sets, in Common Data Format.
\item[DETAILED] Detailed catalogs for the individual data sets and
sensors.
\item[DOCUMENT] Overall documentation for the CD-ROM.
\item[INDEX] Various comma-delimited tables of data describing the
data sets.  Suitable for import into a database management system.
\item[SOFTWARE] Utility programs for manipulating data in CDF under
MS-DOS$^{\rm TM}$, DEC VAX$^{\rm TM}$  or UNIX$^{\rm TM}$ operating
systems, along with the portable CDF software library.
\item[SUMMARY] Summary documentation for the individual data sets.
\end{description}

There are several files associated with each data set.  The kind of
information stored in a file can be inferred from the extension on the
file name:

\begin{description}
\item[CDF] The data, formatted to CDF v2.1 as network-encoded,
single file data sets.  These are all located in the DATA subdirectory.
\item[DET] A text file, containing the NCDS Detailed Catalog
associated with the data set.  All of these files are located in the
DETAILED subdirectory.
\item[SUM] A text file, containing summary information about the
data set.  All of these files are located in the SUMMARY subdirectory.
\end{description}

Since several data sets may share a single detailed catalog, the file
PRODUCTS.LIS (in the SUMMARY subdirectory) gives a tabular listing of
the name of the data file, a brief description of the data set and the
name of the associated detailed catalog.

\section*{Software}

A software package to assist the user in browsing the contents of the
disk and in reading the supplemental documentation is provided here.
Through the use of a commercial user-interface library (JYACC
Application Manager --- JAM), we were able to develop versions of the
software which present the same user interface under MS-DOS, Unix and
VMS.  The interface itself was designed to look and perform like the
current NCDS on-line system.

Explicit installation and operation directions for the browse software
are located in the SOFTWARE subdirectory.


%Arch:::Major new section

\section*{GEDEX Research}

It is hoped that through this consolidation and documentation of
existing data sets, ambiguities and uncertainties associated with
climate change and greenhouse gas effect will be further explored by
more scientists.  It is also hoped that researchers will continue to
review the relationships between temperature change and plausible
cause-effect factors, and that this disk will serve as a test-bed for
future CD-ROMs for EOS.

\section*{Acknowledgements}

The creation and distribution of the GEDEX CD-ROM disk set was an effort
that pressed the entire NCDS staff into service to meet the standards
adopted at the outset. Bruce Vollmer, NCDS task leader, coordinated the
effort for Hughes STX. Ke Jun Sun coaxed CDFs from many nebulously
formatted, vaguely documented, but vitally important data sets.
Configuration management was expertly handled by systems architect John
Vanderpool, who also created the package of software for users of this
disk and helped in data set implementation. The integrity of the data
base, now fully normalized by data base designer Hank Griffioen,
permitted the assimilation of designated data holdings for this CD-ROM.
Staff members Jim Closs and Frank Corprew balanced their workload in the
user support office while managing to implement data sets and help in
data documentation. Overseeing the archiving and inventory of data was
Joe Brown, aided by Frances Bergman.  Rick Amick fine-tuned the artwork,
and helped with editing. Sue Sorlie rendered important assistance in
data verification. Pat Hrubiak, designed the logo for this CD-ROM,
contributing in the areas of quality control and catalog updates as
well.

In addition to the full cooperation from the data set producers already
mentioned, valuable assistance was also offered by Alison Walker and
Helene Wilson from GISS, Mike Rowland and Er-won Chiou from Langley
Research Center,  Lena Iredell from Goddard Space Flight Center, and
Pavel Groisman from the State Hydrological Institute (presently at
NCDC).  Michael J.~Martin of the Jet Propulsion Laboratory provided
valuable assistance by producing a preliminary version of this disk for
testing.

Support for this effort from the Earth Science and Applications
Division, NASA Headquarters was provided by Dr.~Robert Schiffer.
Dr.~Sushel Unninayar (NASA Headquarters/University Corporation for
Atmospheric Research) provided important suggestions and encouragement
throughout.

\bigskip
\noindent
Further information on GEDEX can be obtained by contacting:

\smallskip
\noindent
Dr.~Robert A.~Schiffer \\
Chief, Climate and Hydrological Systems Branch \\
Earth Science and Applications Division \\
NASA Headquarters, Code SED \\
Washington, DC 20546

\noindent
tel: (202) 453-1680 \\
fax: (202) 755-5032

\bigskip
\noindent
Further information on the CD-ROM can be obtained by contacting:

\medskip
\noindent
Lola M.~Olsen \\
Project Manager, NASA's Climate Data System \\
Data Management Systems Facility \\
Goddard Space Flight Center \\
Greenbelt, MD 20771

\noindent
tel: (301) 286-9760 \\
fax: (301) 286-3221

\end{document}


%Arch:  move phone numbers beside our names if this drags on to another
%page
