\begin{ingrid}
continuedataset:
datasetdefs:
/CPTECfilenames
{
(/Data/data1/iri/modelling/daved/WGSIP.D/SampleData.D/CPTEC.D/%s[season].D/GPOSNMC%Y%m[starttime]1%d[M]12%Y%m[T]0112P.mms.T062L28)
[M L S S 25 sub /name /starttime def season T]
} def
:datasetdefs
/description (Centro de Previsão de Tempo e Estudos Climáticos) def
/iridl:hasSemantics (iridl:CPTEC) def
grid:
/name (longitude)def
/units (degree_east)def
 0.0 1.875 360 1 index sub
:grid
grid:
/name (latitude)def
/units (degree_north)def
values:
-88.57217 -86.72253 -84.86197 -82.99894 -81.13498 -79.27056 -77.40589 -75.54106
 -73.67613 -71.81113 -69.94608 -68.08099 -66.21587 -64.35073 -62.48557 -60.62040
 -58.75521 -56.89001 -55.02481 -53.15960 -51.29438 -49.42915 -47.56393 -45.69869
 -43.83346 -41.96822 -40.10298 -38.23774 -36.37249 -34.50724 -32.64199 -30.77674
 -28.91149 -27.04624 -25.18099 -23.31573 -21.45048 -19.58522 -17.71996 -15.85470
 -13.98945 -12.12419 -10.25893  -8.39367  -6.52841  -4.66315  -2.79789  -0.93263
   0.93263   2.79789   4.66315   6.52841   8.39367  10.25893  12.12419  13.98945
  15.85470  17.71996  19.58522  21.45048  23.31573  25.18099  27.04624  28.91149
  30.77674  32.64199  34.50724  36.37249  38.23774  40.10298  41.96822  43.83346
  45.69869  47.56393  49.42915  51.29438  53.15960  55.02481  56.89001  58.75521
  60.62040  62.48557  64.35073  66.21587  68.08099  69.94608  71.81113  73.67613
  75.54106  77.40589  79.27056  81.13498  82.99894  84.86197  86.72253  88.57217
:values
:grid
latitude high low subgrid name exch def
grid:
/name /Z def
/units (mb)def
values:  1000.  850  700  500  300  200. :values
:grid
grid:
/name /M def
/long_name (members) def
2 1 7
:grid
grid:
/name /S def
monthtime
/pointwidth 0 def
periodic
1 Mar 1979 ensotime
3.
1 Dec 1979 ensotime
:grid
variable:
/name /season def
grids: S :grids
values:
(MAMJJA)
(JJASON)
(SONDJF)
(DJFMAM)
:values
:variable
grid:
/name /S def
/long_name (start time) def
monthtime
/pointwidth 0 def
1 Mar 1979 ensotime
3.
1 Dec 2000 ensotime
:grid
grid:
/name /L def
/long_name (lead time) def
/units (months) def
0.5 1. 5.5
:grid
/NVZ 30 NewIntegerGRID name exch def
/T { S L add /long_name (Forecast Time) def /pointwidth 1 def} defasvar
[longitude latitude | NVZ M S L]CPTECfilenames readdatafile:
sequential
/missing_value
-2.56E+33
def
:readdatafile
definevariables:
0 Zvariable:
/name /TOPO def
/units (M) def
/long_name (topography) def
:Zvariable
0 Zvariable:
/name /LSMK def
/long_name (land mask) def
:Zvariable
6 Zvariable:
/name /ua def
/long_name (eastward wind) def
/units (m/s) def
:Zvariable
6 Zvariable:
/name /va def
/long_name (northward wind) def
/units (m/s) def
:Zvariable
6 Zvariable:
/name /zg def
/long_name(time mean geopotential height) def
/units (m) def
:Zvariable
0 Zvariable:
/name /psl def
/long_name (time mean sea-level pressure) def
/units (mb above 1000) def
:Zvariable
6 Zvariable:
/name /ta def
/long_name (air temperature) def
/units (Kelvin_scale) def
:Zvariable
0 Zvariable:
/name /pr def
/long_name (total precipitation) def
/units                      (Kg m-2 day-1) def
:Zvariable
0 Zvariable:
/name /ts def
/long_name (ground/surface cover temperature) def
/units (Kelvin_scale) def
:Zvariable
0 Zvariable:
/name /tcas def
/long_name (temperature of canopy air space) def
/units (Kelvin_scale) def
/iridl:hasSemantics (iridl:LandSurfaceTemperature) def
:Zvariable
:definevariables
:dataset
 {zg ts TOPO ua psl pr tcas LSMK ta va}{exec cachevar pop}forall
\end{ingrid}
Hi Benno,
The CPTEC SMIP2 data is now all here and ready to be put into the Data Library. I will outline here how the
data is divided up. Please let me know if there is something that I can do to make the data easier for you
to incorporate into the Data Library.
The data is in the form of GrADS control file and associated data set with one pair of files for
each ensemble member and each verification month. The labeling is:

he file name is in the format GPOSNMCYYYYMMDDHHYYYYMM0112P.
The first label (from the left) YYYYMMDDHH is the year, month, day and hour of the initial condition. The second label YYYYMM0112 represents the year and month of prediction. 0112 is meaningless, just to complete the correct number of characters in the string. The letter "P" means post-processed.

So, for instance, for the 6 ensemble members that verify in November  1980 the control files would be:

GPOSNMC19781112121980110112P.mms.T062L28.ctl
GPOSNMC19781113121980110112P.mms.T062L28.ctl
GPOSNMC19781114121980110112P.mms.T062L28.ctl
GPOSNMC19781115121980110112P.mms.T062L28.ctl
GPOSNMC19781116121980110112P.mms.T062L28.ctl
GPOSNMC19781117121980110112P.mms.T062L28.ctl

The part of the file:
GPOSNMC1978111212 basically just denotes the different ensemble members.
These cases are the so-called DJFMAM cases and are located in:
/Data/data1/iri/modelling/daved/WGSIP.D/SampleData.D/CPTEC.D/DJFMAM.D

The other "seasons" are located in:

/Data/data1/iri/modelling/daved/WGSIP.D/SampleData.D/CPTEC.D/JJASON.D
/Data/data1/iri/modelling/daved/WGSIP.D/SampleData.D/CPTEC.D/MAMJJA.D
/Data/data1/iri/modelling/daved/WGSIP.D/SampleData.D/CPTEC.D/SONDJF.D

I attach a sample control file which shows the fields.

It would seem to me that I could make things easier for you by eliminating this rather long ensemble member
prefix and putting all months for a particular forecast into a single data/control file pair. Please let me know if that would be
helpful.

Dave




-- 
Dr. David G. DeWitt
223 Monell Building
IRI - International Research Institute for Climate Prediction (IRI)
Lamont-Doherty Earth Observatory
Columbia University
61 Route 9W
Palisades, NY 10964-8000
United States
Phone: (845) 680-4415
Fax: (845) 680-4865
daved@iri.columbia.edu



DSET ^GPOSNMC19981117122001050112P.mms.T062L28
*
OPTIONS SEQUENTIAL YREV BIG_ENDIAN
*
UNDEF -2.56E+33
*
TITLE PRESSURE HISTORY    CPTEC AGCM R1.2 2001  T062L28  warm
*
XDEF 192 LINEAR    0.000   1.8750000000
YDEF  96 LEVELS
 -88.57217 -86.72253 -84.86197 -82.99894 -81.13498 -79.27056 -77.40589 -75.54106
 -73.67613 -71.81113 -69.94608 -68.08099 -66.21587 -64.35073 -62.48557 -60.62040
 -58.75521 -56.89001 -55.02481 -53.15960 -51.29438 -49.42915 -47.56393 -45.69869
 -43.83346 -41.96822 -40.10298 -38.23774 -36.37249 -34.50724 -32.64199 -30.77674
 -28.91149 -27.04624 -25.18099 -23.31573 -21.45048 -19.58522 -17.71996 -15.85470
 -13.98945 -12.12419 -10.25893  -8.39367  -6.52841  -4.66315  -2.79789  -0.93263
   0.93263   2.79789   4.66315   6.52841   8.39367  10.25893  12.12419  13.98945
  15.85470  17.71996  19.58522  21.45048  23.31573  25.18099  27.04624  28.91149
  30.77674  32.64199  34.50724  36.37249  38.23774  40.10298  41.96822  43.83346
  45.69869  47.56393  49.42915  51.29438  53.15960  55.02481  56.89001  58.75521
  60.62040  62.48557  64.35073  66.21587  68.08099  69.94608  71.81113  73.67613
  75.54106  77.40589  79.27056  81.13498  82.99894  84.86197  86.72253  88.57217
ZDEF   6 LEVELS
         1000  850  700  500  300  200
TDEF 1 LINEAR 12z01may2001 1MO
*
VARS  10
TOPO  0 99 TOPOGRAPHY                              (M               )
LSMK  0 99 LAND SEA MASK                           (NO DIM          )
UVMT  6 99 TIME MEAN ZONAL WIND (U)                (M/Sec           )
VVMT  6 99 TIME MEAN MERIDIONAL WIND (V)           (M/Sec           )
ZHMT  6 99 TIME MEAN GEOPOTENTIAL HEIGHT           (M               )
MPMT  0 99 TIME MEAN SEA LEVEL PRESSURE            (Mb-1000         )
ATMT  6 99 TIME MEAN ABSOLUTE TEMPERATURE          (K               )
PREC  0 99 TOTAL PRECIPITATION                     (Kg M**-2 Day**-1)
TGSC  0 99 GROUND/SURFACE COVER TEMPERATURE        (K               )
TCAS  0 99 TEMPERATURE OF CANOPY AIR SPACE         (K               )
ENDVARS
