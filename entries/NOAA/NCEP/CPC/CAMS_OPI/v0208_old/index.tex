\begin{ingrid}
continuedataset:
datasetdefs:
/mydata
{
/D
/character unordered 
  { /gprcp /gcnt /oprcp /prcp /intvar /aprcp /percentofgamma } cvlit
 NewGRID
T 1 7 twogridmodcounter
T (/Data/data5/noaa/cpc/CAMS-OPI-v0208_old/cams_opi_merged.%Y%m) strftimes
copyachunk
[X Y ] -999 readr4
}
defasvarsilent
:datasetdefs
/dataset_documentation.html currentobject 1 index get def
link:
/name (v0208) def
/href (http://iridl.ldeo.columbia.edu/SOURCES/.NOAA/.NCEP/.CPC/.CAMS_OPI/.v0208/) def
/description (this dataset was replaced by v0208) def
/semantics /dcterm:isReplacedBy def
:link
grid:
/name /X def
degE
1.25 2.5 358.75 
periodic
:grid
grid:
/name /Y def
degN
-88.75 2.5 88.75
:grid
grid:
/name /T def
monthtime
16 Jan 1979 ensotime
1.
systemtime 
2419200 sub
16 exch (%b %Y) 10 strftime interp ensotime
/defaultvalue {last} def
:grid
T 
{ (/Data/data5/noaa/cpc/CAMS-OPI-v0208_old/cams_opi_merged.%Y%m[T])[T]} 
removeextrausing 
%last
%rm19602ymd
%ymd:nextmonth
%ymd:nextmonth
% ymd2systemtime
%/expires exch def
name exch def
/last_modified T .last_modified def
%/expires T .expires def
dataset:
/name /percentofgamma def
/prcp 
{
dataset .mydata
D /percentofgamma VALUE D removeGRID
/long_name (precipitation) def
/units (percent) def
} defasvarsilent
:dataset
dataset: 
/name /gauge def
/prcp 
{
dataset .mydata
D /gprcp VALUE D removeGRID
/long_name (precipitation) def
/units (mm/day) def
startcolormap
DATA 0 20 RANGE
white
grey
cyan 0 VALUE
blue 15 VALUE
navy 20 VALUE
darkgrey
endcolormap
} defasvarsilent
/count 
{dataset .mydata
D /gcnt VALUE D removeGRID
} defasvarsilent
:dataset
dataset: 
/name /satellite def
/prcp {
dataset .mydata
D /oprcp VALUE D removeGRID
/long_name (precipitation) def
/units (mm/day) def
startcolormap
DATA 0 20 RANGE
white
grey
cyan 0 VALUE
blue 15 VALUE
navy 20 VALUE
darkgrey
endcolormap
} defasvarsilent
:dataset
dataset:
/name /mean def
/prcp {
dataset .mydata
D /prcp VALUE D removeGRID
/long_name (precipitation) def
/units (mm/day) def
startcolormap
DATA 0 20 RANGE
white
grey
cyan 0 VALUE
blue 15 VALUE
navy 20 VALUE
darkgrey
endcolormap
} defasvarsilent
:dataset
dataset:
/name /anomaly def
/prcp {
dataset .mydata
D /aprcp VALUE D removeGRID
/long_name (precipitation) def
/units (mm/day) def
startcolormap
DATA -40 40 RANGE
white
darkgrey
DarkRed -40 VALUE
red -30 VALUE
orange -20 VALUE
yellow -10 VALUE
moccasin 0 VALUE
PaleGreen 10 VALUE
aquamarine 20 VALUE
DeepSkyBlue 30 VALUE
blue 40 VALUE
navy
endcolormap
} defasvarsilent
:dataset
:dataset
\end{ingrid}
Hi Benno,

I have placed the data files available for the latest version of
CAMS-OPI in the directory /Data/data5/noaa/cpc/CAMS-OPI-v0208 , along
with the documentation and the renamed "gauge.input" file.  At the ncep
ftp server (ftp.ncep.noaa.gov) the files are in the directory
/pub/precip/data-req/cams_opi_v0208 .

An excerpt from the documentation:

The "CAMS_OPI" data are 2.5 degree lat/lon spatial means of monthly
mean precipitation derived from station gauges and satellite estimates.
Each data 'record' is a 144 x 72 array, oriented from SOUTH -> North
and Eastward from 1.25E.

The array orientation is:


     0     2.5E   5.0E        2.5W     0
     |      |      |           |       |
     -----------------------------------  South Pole
     |      |      |           |       |
     |(1,1) | (2,1)|   ....    |(144,1)|.............88.75S
     |      |      |           |       |
     -----------------------------------  87.5S
     |      |      |           |       |
     |(1,2) | (2,2)|   ....    |(144,2)|.............86.25S
     |      |      |           |       |
     -----------------------------------  85.0S
                        .
                        .
                        .
     -----------------------------------  87.5N
     |      |      |           |        |
     |(1,72)|(2,72)|   ....    |(144,72)|............88.75N
     |      |      |           |        |
     -----------------------------------   Nort Pole
         .      .                  .
         .      .                  .
         .      .                  .
       1.25E  3.75E               1.25W

     Note that the the coordinates for each grid correspond to the
center of
     each box since the data values are spatial averages.

A separate data file for each month exists.  The data arrays are in a
FORTRAN
77 "direct", binary records ("IEEE") and are grouped as follows:

  Record 1: Precipitation analysis based on raingauge data only (CAMS)
               2: number of CAMS gauges
               3: Precipitation analysis based on OPI estimates only
               4: The blended analysis CAMS_OPI monthly total precip.
analysis
               5: For Internal Use only
               6: CAMS_OPI monthly anomaly (1979-1995 base period)
               7:    "        "       "    expressed as % of gamma
distribution

The units are "mm/day" for the means and anomalies.


  --Michael
