\documentstyle[code,ingrid]{article}
\topmargin -0.5in        % read Lamport p.163
\oddsidemargin -0.04cm   % read Lamport p.163
\evensidemargin -0.04cm  % same as oddsidemargin but for left-hand pages
\textwidth 6.5in
\textheight 8.5in
%\pagestyle{empty}       % Uncomment if don't want page numbers
\parskip 7.2pt           % sets spacing between paragraphs
%\renewcommand{\baselinestretch}{1.5} 	% Uncomment for 1.5 spacing between lines
\parindent 0pt		  % sets leading space for paragraphs

\begin{document}
\section{CPC Merged Analysis of Precipitation Dataset Entry}
\begin{ingrid}
/long_name (November 2016 Release) def
continuedataset:
\end{ingrid}
\subsection{hidden function definitions}
\begin{ingrid}
 datasetdefs:
\end{ingrid}
\word prcp\_mmperday\_colors ( -- ) precipitation color scale in units of mm/day.
\begin{ingrid}
/prcp_mmperday_colors {
 startcolormap
 DATA 0 22 RANGE
 white white grey 0 VALUE
 grey grey 0.1 bandmax
 210 255 215 RGB RGBdup 2 bandmax
 190 240 195 RGB RGBdup 4 bandmax
 150 225 155 RGB RGBdup 6 bandmax
 115 210 110 RGB RGBdup 8 bandmax
 100 195 105 RGB RGBdup 10 bandmax
  90 180  90 RGB RGBdup 12 bandmax
  80 165  80 RGB RGBdup 14 bandmax
  70 150  70 RGB RGBdup 16 bandmax
  60 135  60 RGB RGBdup 18 bandmax
  50 120  50 RGB RGBdup 20 bandmax
  40 105  40 RGB RGBdup 22 bandmax
  30  90  30 RGB endcolormap } def
 :datasetdefs
\end{ingrid}
\subsection{grid definitions}
\word X ( -- grid ) longitude grid.
\begin{ingrid}
grid:
/name /X def
/units (degreeE) def
/fullname (Longitude) def
/standard_name (longitude) def
periodic
1.25 2.5 358.75
:grid
\end{ingrid}
\word Y ( -- grid ) latitude grid.
\begin{ingrid}
grid:
/name /Y def
/units (degreeN) def
/fullname (Latitude) def
/standard_name (latitude) def
-88.75 2.5 88.75
:grid
\end{ingrid}
\word T ( -- grid ) time in months.
\begin{ingrid}
grid:
/name /T def
monthtime
/fullname (Month) def
/standard_name (time) def
/defaultvalue { last } def
/pointwidth 1 def
16 Jan 1979 ensotime
1.
16 Oct 2016 ensotime %enddate
:grid

dataset:
\end{ingrid}
Copy X, Y, and T grids into CMAP version 1 dataset
\begin{ingrid}

X name exch def
Y name exch def
T name exch def

\end{ingrid}
Specify name, long\_name and description for CMAP version 1 dataset
\begin{ingrid}
 /name /ver1 def
 /long_name (Version 1) def
 /description (using rain gauge, satellite, and numerical model predictions) def

\end{ingrid}
\subsection{dependent variable definitions}
The data for this dataset are held in text files -- one for each year of available data.
Two variables are held in each text file.  Use the NVZ grid, which has two points, to index
the two variables when reading data from the file.  Using the monthly T grid, apply a set of
functions, "yearlyedgesgrid first secondtolast subgrid" to get the four-digit year values
(interpolated in the file name using "\%y[T]") to cleanly generate the names of the yearly files 
from which to read the monthly data.
\begin{ingrid}

 [ /NVZ 2 NewIntegerGRID X Y | T ] 
(/Data/data5/noaa/cac/cmap/monthly/v1611/cmap_mon_v1611_%y[T].txt) 
[T yearlyedgesgrid first secondtolast subgrid] readdatafile: 
((24X,2F8.2))formatted
  /missing_value -999.0 def
:readdatafile

definevariables:

\end{ingrid}
Assign attributes to the two variables read from the data files
\word prcp\_est ( -- var ) estimated precipitation.
\begin{ingrid}
 
0 Zvariable:
/name (prcp_est) def
/long_name (CMAP Estimated Precipitation) def
/units (mm/day) def
prcp_mmperday_colors
/standard_name (lwe_precipitation_rate) def
:Zvariable

\end{ingrid}
\word error ( -- var ) estimated error of precipitation estimate.
\begin{ingrid}
0 Zvariable:
/name /error def
/units (percent) def
:Zvariable

:definevariables

:dataset

dataset:

\end{ingrid}
Copy X, Y, and T grids into CMAP version 2 dataset
\begin{ingrid}
X name exch def
Y name exch def
T name exch def

\end{ingrid}
Specify name, long\_name and description for CMAP version 2 dataset
\begin{ingrid}
 /name /ver2 def
 /long_name (Version 2) def
 /description (using rain gauge and satellite estimates) def

\end{ingrid}

\subsection{dependent variable definitions}
The data for this dataset are held in text files -- one for each year of available data.
Two variables are held in each text file.  Use the NVZ grid, which has two points, to index
the two variables when reading data from the file.  Using the monthly T grid, apply a set of
functions, "yearlyedgesgrid first secondtolast subgrid" to get the four-digit year values
(interpolated in the file name using "\%y[T]") to cleanly generate the names of the yearly files 
from which to read the monthly data.
\begin{ingrid}
 [ /NVZ 2 NewIntegerGRID X Y | T ] 
(/Data/data5/noaa/cac/cmap/monthly/v1611/cmap_mon_v1611_%y[T].txt) 
[T yearlyedgesgrid first secondtolast subgrid] readdatafile: 
((40X,2F8.2))formatted
  /missing_value -999.0 def
:readdatafile

definevariables:

\end{ingrid}
Assign attributes to the two variables read from the data files
\word prcp\_est ( -- var ) estimated precipitation.
\begin{ingrid}
0 Zvariable:
/name (prcp_est) def
/long_name (CMAP Estimated Precipitation) def
/units (mm/day) def
prcp_mmperday_colors
/standard_name (lwe_precipitation_rate) def
:Zvariable

\end{ingrid}
\word error ( -- var ) estimated error of precipitation estimate.
\begin{ingrid}
0 Zvariable:
/name /error def
/units (percent) def
:Zvariable

:definevariables

:dataset

:dataset

\end{ingrid}
\end{document}
