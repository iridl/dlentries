\begin{ingrid}

continuedataset:

Ingrid:

/rfediff_colors { %starts the definition of a colorscale names  rfediff_colors.
startcolormap -150 150 RANGE
DimGray sienna
sienna -150 VALUE
sienna sienna -100 bandmax
peru peru -50 bandmax
burlywood burlywood -25 bandmax
wheat wheat -5 bandmax
LightYellow LightYellow 5 bandmax
LightCyan LightCyan 25 bandmax
PaleTurquoise PaleTurquoise 50 bandmax
MediumAquamarine MediumAquamarine 100 bandmax
CadetBlue CadetBlue 150 bandmax
CadetBlue
endcolormap
}defcolorscale %ends the definition of colorscales

%the same for another colorscale

/rfepercent_colors {
startcolormap
0 300 RANGE
DimGray white SaddleBrown
0 VALUE
SaddleBrown SaddleBrown
25 bandmax
peru peru
50 bandmax
burlywood burlywood
75 bandmax
wheat wheat
100 bandmax
LightCyan LightCyan
120 bandmax
aquamarine aquamarine 140
bandmax
MediumAquamarine MediumAquamarine 160
bandmax
MediumSeaGreen MediumSeaGreen 180
bandmax
SeaGreen SeaGreen 200
bandmax
DarkSeaGreen DarkSeaGreen 300
bandmax
DarkSeaGreen endcolormap
}defcolorscale

pop %ends the definition of colorscales

%datasetdefs:
/T
1 Jan 1981 julian_day
31 Jan 2022 julian_day
% systemtime (%d %b %Y) 12 strftime interp julian_day
dekadGRID
/defaultvalue { last } def
def
%:datasetdefs

 % T
 %  { (/Data/data23/Vietnam/ENACTS_MERGED/MON/PRECIP/DEKAD/rr_mrg_%Y[T]%m[T]%d[dekad]_MON.nc) [T dup day2dom 5. add 10. div toi4 /name /dekad def]}removeextrausing
 %  /expires last_modified systemtime2ymd dup
 %       20 lt
 %      ymd2systemtime
 %    def
 %  name exch def
 % /expires T .expires def
 % /last_modified T .last_modified def

grid:
/name /Y def
/units /degree_north def
8.025  0.05 23.475
/long_name (Latitude) def
:grid

grid:
/name /X def
/units /degree_east def
102.025  0.05 109.475
/long_name (Longitude) def
:grid

 /rfe_merged{
[ X Y | T] (/Data/data23/Vietnam/ENACTS_MERGED/MON/PRECIP/DEKAD/rr_mrg_%Y[T]%m[T]%d[dekad]_MON.nc)
[T dup day2dom 5. add 10. div toi4 /name /dekad def ]
readdatafile:
/name (precip) def
/long_name (Merged Station-Satellite Rainfall) def
/units /mm def
/missing_value -99. def
/realarraytype netcdfrecords
:readdatafile
%cmorph_dekad_colors %applies a colorscale named cmorph_dekad_colors to the variable
precip_colors
%DATA 0 600 RANGE
}defasvarsilent

%This section defines variables that are computed from the variables that were read from datafiles. They don't read any actual datafiles, they are pure Ingrid code.

/SPI-rfe_merged_1-dekad { %starts a variable and names it
rfe_merged                %selects the rfe_merged precipitation variable
  1 36 gamma3par          %starts the computation of SPI for dekadal precipitation variable
  pcpn_accum gmean gsd gskew pzero 1 gammaprobs
  1 gammastandardize      %ends computation SPI for dekadal precipitation variable
  /long_name (Standard Precipitation Index 1-dekad) def %gives the variable a more understandable name to show up in maps and graphs
}defasvarsilent           %ends the definition of the variables

/rfe_mergeddiff {         %starts a variable and names it
  rfe_merged              %selects rfe_merged variable
  SOURCES .Vietnam .ENACTS .MON .dekadal .climatologies .rfe_merged %selects dekadal climatology
   T 2 index .T           %starts the computation of dekadal anomalies from dekadal climatology
     a: .first cvsunits
       :a: .last cvsunits :a
      RANGE
   T 2 index .T replaceGRID
   sub                    %ends the computation of dekadal anomalies from dekadal climatology
%Note that you can use those few lines for any dekadal data for which you have a dekadal climatology

%startcolormap -150 150 RANGE %starts the definition of a new colorscale for this variable
%DimGray sienna
%sienna -150 VALUE
%sienna sienna -100 bandmax
%peru peru -50 bandmax
%burlywood burlywood -25 bandmax
%wheat wheat -5 bandmax
%LightYellow LightYellow 5 bandmax
%LightCyan LightCyan 25 bandmax
%PaleTurquoise PaleTurquoise 50 bandmax
%MediumAquamarine MediumAquamarine 100 bandmax
%CadetBlue CadetBlue 150 bandmax
%CadetBlue
%endcolormap % ends the definition of a new colorscale for this variable

/scale_symmetric true def prcp_anomaly DATA AUTO AUTO RANGE

/long_name (Merged Station-Satellite Rainfall Anomaly) def %gives the variable a more descriptive name

}defasvarsilent %ends the variable definiton

%Same as above but computes the relative difference between rfe_adj and its climatology
%Try to copy and paste the code that is within the curly brackets '{' and '}' and see what the code is doing step by step



/rfe_mergeddiff_percent {

  rfe_merged
  SOURCES .Vietnam .ENACTS .MON .dekadal .climatologies .rfe_merged
   T 2 index .T
     a: .first cvsunits
       :a: .last cvsunits :a
      RANGE
   T 2 index .T replaceGRID
   sub
  SOURCES .Vietnam .ENACTS .MON .dekadal .climatologies .rfe_merged
%   10 maskle
   T 2 index .T
     a: .first cvsunits
       :a: .last cvsunits :a
      RANGE
   T 2 index .T replaceGRID
   div
  100 mul
%  -1000 replaceNaN
/units /percent def
{Below_Normal 80 Normal 120 Above_Normal} classify
[rfe_merged]dominant_class
startcolormap
transparent sienna sienna  white DarkGreen DarkGreen
%transparent red red white DarkGreen DarkGreen
endcolormap
 /long_name (Merged Station-Satellite Rainfall Difference Expressed as Percent of Long-Term Average (%)) def
}defasvar


/rfe_merged_vs_meanrfe_merged {
  rfe_merged
  SOURCES .Vietnam .ENACTS .MON .dekadal .climatologies .rfe_merged
   7 maskle
   T 2 index .T
     a: .first cvsunits
       :a: .last cvsunits :a
      RANGE
   T 2 index .T replaceGRID
   div
  100 mul
  -1000 replaceNaN
rfepercent_colors
  /long_name (Merged Station-Satellite Rainfall Expressed as Percentages of the Long-Term Average) def
}defasvar

:dataset

\end{ingrid}
