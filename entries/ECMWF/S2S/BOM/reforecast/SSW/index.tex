\begin{ingrid}
continuedataset:

\end{ingrid}
%This definition is needed for making specific time steps, first, we create an S grid standard then modify it for the right time step
grid:
/name (S) def
/units (days since 1960-01-01) def
/long_name (forecast start time) def
/standard_name (forecast_reference_time) def
01 Jan 2014 julian_day 01 Jan 1960 julian_day sub
1.0
systemtime
(%d %b %Y) 12 strftime interp julian_day
01 Jan 1960 julian_day sub
/pointwidth 0. def
/defaultvalue { last } def
:grid

\begin{ingrid}


/S /julian_day 0 26
 Dec 2014 julian_day
 mark 01
 Jan 2014 julian_day
 null { dup jd2dmy 3 -1 roll 5 add dup 31 ge {pop exch 1 add dup 11 gt {pop 1 add 0}if exch 1} if 3 1 roll exch .5 add exch julian_day dup counttomark 1 add index gt {leave}if}repeat counttomark integerarray astore
 nip nip NewGRID
 /add_offset -.5 def
 (days since 1960-01-01) gridunitconvert
 /pointwidth 0. def
 first secondtolast subgrid
 /long_name (forecast start time) def
 /standard_name (forecast_reference_time) def
name exch def

grid:
/name /hdate def
/units (months since 1960-01-01) def
01 Jul 1981 ensotime
12.0
01 Jul 2013 ensotime
/defaultvalue { last } def
:grid

grid:
/name /L def
/long_name (Lead) def
/standard_name (forecast_period) def
/units (days) def
0. 1. 32.
/pointwidth 0. def
:grid

grid:
/name /M def
/long_name (Ensemble Member) def
/standard_name (realization) def
/units (ids) def
1 1 62
:grid



 /ensembles { [ M L | S hdate ] 
 (/Data/data25/ECMWF/S2S/BOM/REF/SSW/%Y[S]/%m[S]/SSW.bom.%Y[hdate]%m%d[S].20140101) [ S hdate ]
readdatafile:
((32(62(26X,F6.7,/)),(61(26X,F6.7,/),(26X,F6.7)))) formatted
:readdatafile
/units (unitless) def
} defasvarsilent



:dataset
\end{ingrid}
