\begin{ingrid}
/description (Dekadal NDVIg data) def
continuedataset:
datasetdefs:
/filename {T (/Data/data6/usgs/adds/ndvi/ndvig/ndvig%y[T]/%y%m0%d[dekad].bil) strftimes}def
/adds_ndvi_colors {
startcolormap
DATA 0.0 1.0 RANGE
white white 0.0 VALUE
234 233 189 RGB RGBdup 0.05 bandmax
215 211 148 RGB RGBdup 0.1 bandmax
202 189 119 RGB RGBdup 0.15 bandmax
175 175 77 RGB RGBdup 0.2 bandmax
128 169 5 RGB RGBdup 0.3 bandmax
12 127 0 RGB RGBdup 0.4 bandmax
0 94 0 RGB RGBdup 0.5 bandmax
0 59 1 RGB RGBdup 0.6 bandmax
0 9 0 RGB RGBdup 1.0 bandmax
0 9 0 RGB 1.0 VALUE
0 9 0 RGB
endcolormap } def
:datasetdefs
/Conventions (CF-1.0) def
grid:
/name /x def
-4604000 8000 4604000
/standard_name (projection_x_coordinate) def
/units (m) def
:grid
grid:
/name /y def
4604000 8000 -4604000
/standard_name (projection_y_coordinate) def
/units (m) def
:grid
variable:
        /name /lat def
        /units (degreeN) def
        /fullname (Latitude) def
	/standard_name (latitude) def
        /scale_min -40 def
        /scale_max 45 def
        grids: x y :grids
        file:
                decrandom
                /name (/Data/data6/usgs/adds/ndvi/lat_1152_geom.bin) def
        :file
:variable
variable:
        /name /lon def
        /units (degreeE) def
        /fullname (Longitude) def
	/standard_name (longitude) def
        /scale_min -24 def
        /scale_max 65 def
        grids: x y :grids
        file:
                decrandom
                /name (/Data/data6/usgs/adds/ndvi/lon_1152_geom.bin) def
        :file
:variable

1 Jul 1981 julian_day
%systemtime (%d %b %Y) 12 strftime interp julian_day
%26 May 2004 julian_day %enddate
26 Dec 2008 julian_day %enddate
dekadGRID
/defaultvalue { last } def
/standard_name (time) def
name exch def

/NDVI {[ x y | T] (/Data/data6/usgs/adds/ndvi/ndvig/ndvig%y[T]/%y[T]%m[T]0%d[dekad].bil) [T dup day2dom 5. add 10. div toi4 /name /dekad def ]
readdatafile:
%readui1direct
ui*1
:readdatafile
/valid_min 1 def
/valid_max 250 def
/scale_factor 0.004016 def
/add_offset -0.004016 def
/grid_mapping /Projection def
/long_name (Normalized Difference Vegetation Index) def
/standard_name (normalized_difference_vegetation_index) def
/iridl:hasSemantics (iridl:NDVI) def
/SpatialReferenceSystemDims {x y} cvlit def
/SpatialReferenceSystemWKT (PROJCS["Albers_Equal_Area_Conic",GEOGCS["GCS_North_American_1927",DATUM["D_North_American_1927",SPHEROID["Clarke_1866",6378206.4,294.9786982]],PRIMEM["Greenwich",0],UNIT["Degree",0.017453292519943295]],PROJECTION["Albers"],PARAMETER["False_Easting",0],PARAMETER["False_Northing",0],PARAMETER["Central_Meridian",20],PARAMETER["Standard_Parallel_1",21],PARAMETER["Standard_Parallel_2",-19],PARAMETER["Latitude_Of_Origin",1],UNIT["Meter",1]]) def
toNaN
adds_ndvi_colors
x /standard_name (projection_x_coordinate) def name exch def
y /standard_name (projection_y_coordinate) def name exch def
}defasvarsilent

variable:
 grids: :grids
 values: 0 :values
 /name (Projection) def
 /grid_mapping_name (albers_conical_equal_area) def
 /latitude_of_projection_origin 1 def
 /longitude_of_central_meridian 20.0 def
 /standard_parallel 21 -19 2 realarray astore def
 /false_easting 0 def
 /false_northing 0 def
:variable

%/references (Tucker, C. J., J. E. Pinzon, et al., 2005. "An Extended AVHRR 8-km NDVI Data Set Compatible with MODIS and SPOT Vegetation NDVI Data." International Journal of Remote Sensing, submitted.) def

%/references (Pinzon, J., M. E. Brown, et al., 2004.  Satellite time series correction of orbital drift artifacts using empirical mode decomposition.  Hilbert-Huang Transform:  Introduction and Applications.  N. Huang:  Chapter 10, Part II.  Applications.) def

:dataset
%T { (/Data/data6/usgs/adds/ndvi/ndvig/ndvig%y[T]/%y[T]%m[T]0%d[dekad].bil) [T dup day2dom 5. add 10. div toi4 /name /dekad def]}removeextrausing
%name exch def
\end{ingrid}
