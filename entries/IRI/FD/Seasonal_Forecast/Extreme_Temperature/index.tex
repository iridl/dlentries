\documentclass[12pt]{article}
\usepackage{code}
\usepackage{ingrid}
\begin{document}
\section{IRI Extreme Temperature Dataset Entry}
This is the dataset for the extreme temperature, along with
its anomaly temperature scale.
\subsection{hidden function definitions}
Hidden function definitions in datasetdefs:
\begin{ingrid}

 continuedataset:
/references (Barnston_etal2003 Barnston_Mason2011) def
%Barnston, A.G., S.J. Mason, L. Goddard, D.G. DeWitt, S.E. Zebiak. 2003. Multimodel Ensembling in Seasonal Climate Forecasting at IRI. Bull. Amer. Met. Soc. 84 (12): 1783-1796. DOI: 10.1175/BAMS-84-12-1783
%Barnston, A. G. and S. J. Mason, 2011: Evaluation of IRI's seasonal climate forecasts for the extreme 15% tails. Wea. Forecasting, 26, 545-554. DOI: 10.1175/WAF-D-10-05009.1
 datasetdefs:
\end{ingrid}
\word mytempanomscale2 ( -- ) anomaly temperature scale.  We are using this color scale for temperature extremes.
\begin{ingrid}
/mytempanomscale2 {
startcolormap
DATA -100 100 RANGEEDGES
transparent 0 0 250 RGB
0 0 250 RGB RGBdup -50 bandmax
125 165 250 RGB RGBdup -40 bandmax
200 240 250 RGB RGBdup -25 bandmax
gray RGBdup
25 bandmax
220 220 0 RGB RGBdup 40 bandmax
250 175 0 RGB RGBdup 50 bandmax
240 30 0 RGB RGBdup 100 bandmax
240 30 0 RGB
endcolormap
} def
\end{ingrid}
\word togetherfn2 ( -- ) function used to compute the dominant tercile.  At the moment this is not the right scale.
\begin{ingrid}
/togetherfn2 {
 0 replaceNaN
 a:
    C /Below_Normal VALUE
    -1 mul
    :a:
     C /Above_Normal VALUE
    :a
  add
 mytempanomscale2 
 /missing_value 0 def
 -9 replaceNaN
 /missing_value -9 def
} def      
\end{ingrid}
\word seasabbrev ( -- var ) periodic variable of season abbreviations. 
\begin{ingrid}  
 grid:
 /name (T2) def
 /units (months since 01-Jan) def
 periodic
 0.5 1 11.5
 :grid
variable:
 /name /seasabbrev def
 grids: T2 :grids
 values: (djf) (jfm) (fma) (mam) (amj) (mjj) (jja) (jas) (aso) (son) (ond) (ndj) :values
:variable
 :datasetdefs
\end{ingrid}
\subsection{grid definitions}
\word X ( -- grid ) longitude grid.
\begin{ingrid}
 grid:
 /name (X) def
 /units (degreeE) def
 /fullname (Longitude) def
 periodic
 -179. 2. 179. 
 :grid
\end{ingrid}
\word Y ( -- grid ) latitude grid.
\begin{ingrid}
 grid:
 /name (Y) def
 /units (degreeN) def
 /fullname (Latitude) def
 89. 2. -89.
 :grid
\end{ingrid}
\word L ( -- grid ) lead time in months.
\begin{ingrid}
 grid:
 /name (L) def
 /units (months) def
 /fullname (Forecast Lead Time in Months) def
 values: 1 :values
 :grid
\end{ingrid}
\word F ( -- grid ) forecast start times.  IRI Forecasts have gotten more
frequent over time, so the grid starts out coarser (every three
months) and then is finer (every month).
\begin{ingrid}
 grid:
 /name (F) def
 monthtime
 /fullname (Month Forecast Issued) def
 /defaultvalue { last } def
 /pointwidth 1 def
values:
16 Mar 2001 ensotime
3
16 Jun 2001 ensotime
{} for
16 Sep 2001 ensotime
1
16 Mar 2017 ensotime %enddate
{} for
:values
 :grid
\end{ingrid}
Once the initial version of F is defined using systemtime for the
maximum possible endtime, F is refined via removeextrausing
by checking the data directory for available files, starting from the
end.  Admittedly this is a bit long-winded:  either the last file
should be there, or it isn't. 

removeextrausing is rather critical here -- the data files are being
used both for maximum possible endtime, and for the last\_modified time
for the F grid: that time is taken from the file that was used to set
the maximum possible endtime.

expires computes when to expect the next update, and this dataset we
use a two part strategy. For the first week, it keeps using the
last\_modified time so that any changes to the data are using
immediately, later the ninth of next month is used, with a consequent
boost to the last\_modified time so the change in expires is reported
properly. 
Finally the modified F is stored in the dataset.

copies expires and last\_modified from F to dataset

\word C ( -- grid ) tercile classes.
\begin{ingrid}
 grid:
 /name (C) def
 /units (ids) def
 /fullname (Tercile Classes) def
 values: /Below_Normal /Normal /Above_Normal :values
 :grid
\end{ingrid}
\subsection{dependent variable definitions}
\word target\_date ( -- var ) target date from start and lead.
\begin{ingrid}
 /target_date {F L add 1 add /long_name (forecast time) def /pointwidth 3 def} defasvar
\end{ingrid}
\word target\_season ( -- var ) target three-letter season.
\begin{ingrid}
/target_season {seasabbrev target_date T2 sample-along} defasvar
\end{ingrid}
\word prob ( -- var ) tercile probability.
\begin{ingrid}
/prob {
[ C X Y | L F] (/Data/data1/iriforecasts/T1y%Ys%m[T]%s[target_season]x) [L F 2 copy add 1 add /name /T def target_season] readdatafile:
((13X,3F4.0))formatted
 /missing_value -9 def
 /long_name (Tercile Probability) def
/scale_min 0 def
/scale_max 100 def
 startcolormap
DATA 0 100 RANGE
transparent transparent
transparent transparent 25.0 bandmax
175 238 238 RGB RGBdup 40 bandmax
200 255   0 RGB RGBdup 50 bandmax
176  48  96 RGB RGBdup 100 bandmax
176  48  96 RGB 100 VALUE
176  48  96 RGB
endcolormap
:readdatafile
 /long_name (Tercile Probability) def
} defasvarsilent
\end{ingrid}
\word dominant ( -- var ) dominant tercile probabiliy.
 \begin{ingrid}
/dominant {
 prob togetherfn2
 /long_name (Extreme Temperature Risk) def 
 } defasvarsilent
\end{ingrid}
\begin{ingrid}
 :dataset
\end{ingrid}
\end{document}
